\documentclass[a4paper]{article}
\title{Histogram Equalization}
\author{Chan Chung Yuk, Lo Yu Shing}
\date{\today}
\begin{document}
\maketitle

\section{Description}
In this project, we will extend the auto contrast capability in the image lab by applying Histogram Equalization. 

Histogram Equalization is a process where we uses the cumulative distribution function (c.d.f.) of the histogram of a image to remap it's pixel value such that the resulting histogram of the image is `stretched out' and the contrast of the image improved as a result.

\section{Functionalities}
Here is a non-exhaustive list of functionalities that we will be adding to the lab program
\begin{itemize}
    \item Add our operation as a new option in the Automatic Contrast function.
    \item Implement Histogram Equalization for grayscale histogram and RGB histogram
    \item Visualization of the histogram of the original image and the processed image in the Automatic Contrast function.
    \item Different visualization options for the histogram 
    \begin{itemize}
	\item Toggle for overlaying different color channel's histogram.
	\item Toggle to hide color channel
	\item Toggle to show the c.d.f of the histogram 
    \end{itemize}
\end{itemize}

\end{document}
